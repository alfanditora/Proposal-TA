\begin{table}[H]
  \centering
  \begin{tabular}{ | p{2.5cm} | p{4.75cm} | p{4.75cm} |}
	\hline
	\textbf{Parameter Perbandingan} & \textbf{Sistem \textit{As-Is}} & \textbf{Sistem \textit{To-Be}} \\ \hline
	Pendekatan Dasar & Manual (Manusia), pengguna melakukan inspeksi visual satu per satu. & Otomatis (Algoritma), sistem melakukan pencarian menggunakan \textit{Face Recognition} \\ \hline
    Masukan (\textit{input}) & Hanya berupa akses ke penyimpanan berisi kumpulan foto & Akses penyimpanan ditambah foto \textit{selfie} sebagai referensi \\ \hline
    Mekanisme Proses & Iterasi manual yang diulang sebanyak jumlah foto & Iterasi sistem \\ \hline
    Teknologi & Tidak ada teknologi cerdas & Menggunakan algoritma \textit{face recognition} dengan pendekatan \textit{one-shot learning} \\ \hline
    Peran Pengguna & Aktif dan repetitif & Pasif dan hanya perlu mengunggah data di awal \\ \hline
    Manajemen \textit{Output} & Pemisahan/penandaan foto dilakukan manual oleh pengguna saat itu juga & Sistem secara otomatis menyaring dan menyimpan foto yang cocok ke dalam folder khusus \\ \hline
	\end{tabular}
\caption{Perbandingan Sistem Awal dan Usulan}
\label{tbl:perbandingan sistem}
\end{table}