\begin{table}[H]
  \centering
  \begin{tabular}{ | p{2cm} | p{3cm} | p{7cm} |}
	\hline
	\textbf{Kode} & \textbf{Aspek} & \textbf{Nama Kebutuhan} \\ \hline
	NF1 & \textit{Performance} & \begin{enumerate} \item Sistem harus dapat memproses pencarian pada minimal 50 foto dalam waktu kurang dari 30 detik. \item Akurasi identifikasi foto pribadi harus lebih dari 90\%. \end{enumerate} \\ \hline
	NF2	& \textit{Usability} & \begin{enumerate} \item Antarmuka sistem harus sederhana, intuitif, dan mudah digunakan oleh pengguna. \end{enumerate} \\ \hline
	NF3	& \textit{Security} & \begin{enumerate} \item Sistem harus menjamin keamaan data pribadi yang berupa foto-foto masukan dari pengguna. \end{enumerate} \\ \hline
    NF4 & \textit{Reliability} & \begin{enumerate} \item Sistem harus dapat menangani kesalahan seperti \textit{file} rusak, atau format yang tidak valid. \end{enumerate} \\ \hline
	NF5	& \textit{Scalability} & \begin{enumerate} \item Sistem harus dapat ditingkatkan untuk menangani jumlah foto hingga 100 lebih foto tanpa penurunan performa \end{enumerate} \\ \hline
	NF6	& \textit{Maintainability} & \begin{enumerate} \item Kode sistem harus terdokumentasi dengan baik agar mudah untuk dperbarui. \end{enumerate} \\ \hline
    NF7 & \textit{Portability} & \begin{enumerate} \item Sistem dapat dijalankan pada berbagai sistem operasi. \end{enumerate} \\ \hline
	NF8	& \textit{Availability} & \begin{enumerate} \item Sistem dapat diakses dengan uptime minimal 90\% untuk memastikan layanan selalu tersedia. \end{enumerate} \\ \hline
	\end{tabular}
\caption{Tabel \textit{Non-Functional Requirement}}
\label{tbl:kebutuhan nonfungsional}
\end{table}