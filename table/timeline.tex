\begin{longtable}{@{\extracolsep{\fill}}
    >{\raggedright\arraybackslash}p{2cm}
    >{\raggedright\arraybackslash}p{3cm}
    >{\raggedright\arraybackslash}p{7cm}}
\caption{Timeline Rencana Implementasi}\label{tbl:timeline} \\
\midrule
\textbf{Minggu} & \textbf{Tahap} & \textbf{Kegiatan} \\
\midrule
\endfirsthead

\caption[]{Timeline Rencana Implementasi (lanjutan)} \\
\midrule
\textbf{Minggu} & \textbf{Tahap} & \textbf{Kegiatan} \\
\midrule
\endhead

\midrule
\multicolumn{3}{r}{\textit{Bersambung ke halaman berikutnya}} \\
\endfoot

\midrule
\endlastfoot

1--2 & Persiapan \textit{Dataset} (\textit{Data Preparation}) &
\begin{itemize}
    \item Pengumpulan \textit{dataset} foto untuk skenario pengujian;
    \item Pelabelan (\textit{annotation}) data secara manual untuk membuat \textit{ground truth}.
\end{itemize}
\\

3--5 & Eksplorasi \& Implementasi Algoritma &
\begin{itemize}
    \item Implementasi 5 kategori metode \textit{face representation}: Geometri, Holistik, Berbasis Fitur, Hibrida, dan \textit{Deep Learning};
    \item Pembuatan modul \textit{face detection}, \textit{alignment}, dan \textit{matching};
    \item Pengukuran awal untuk memilih algoritma terbaik berdasarkan kompleksitas model.
\end{itemize}
\\

6--7 & Integrasi Sistem (\textit{System Construction}) &
\begin{itemize}
    \item Pembangunan modul \textit{database} untuk menyimpan \textit{embedding} foto;
    \item Implementasi alur sesuai desain sistem \textit{to-be}.
\end{itemize}
\\

8--9 & Pengujian \& Evaluasi (\textit{Testing}) &
\begin{itemize}
    \item Pengujian performa berdasarkan parameter keberhasilan;
    \item Evaluasi sistem berdasarkan kebutuhan non-fungsional.
\end{itemize}
\\

10--11 & Dokumentasi \& Pelaporan &
\begin{itemize}
    \item Analisis hasil perbandingan antar algoritma;
    \item Penulisan laporan Tugas Akhir lengkap;
    \item Persiapan materi presentasi sidang akhir.
\end{itemize}
\\

\end{longtable}
