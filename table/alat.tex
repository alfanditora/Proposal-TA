\begin{longtable}{@{\extracolsep{\fill}} l p{3cm} p{3cm} p{6cm}}
\caption{Alat dan Bahan Penelitian}\label{tbl:alat_bahan} \\
\midrule
\textbf{No.} & \textbf{Kategori} & \textbf{Alat/Bahan} & \textbf{Keterangan} \\
\midrule
\endfirsthead

\caption[]{Alat dan Bahan Penelitian (lanjutan)} \\
\midrule
\textbf{No.} & \textbf{Kategori} & \textbf{Alat/Bahan} & \textbf{Keterangan} \\
\midrule
\endhead

\midrule
\multicolumn{4}{r}{\textit{Bersambung ke halaman berikutnya}} \\
\endfoot

\midrule
\endlastfoot

1 & Perangkat Keras & Laptop (\textit{Workstation}) &
Laptop dengan spesifikasi CPU Intel Core i7, RAM 16 GB, dan SSD yang digunakan untuk pengembangan kode lokal dan dokumentasi \\

2 & Bahasa Pemrograman & Python 3.8+ &
Bahasa pemrograman utama untuk implementasi sistem, dipilih karena dukungan ekosistem
\textit{deep learning} dan pemrosesan citra yang lengkap \\

3 & Data & NumPy, Pandas &
NumPy digunakan untuk operasi matriks pada data citra, sedangkan Pandas untuk manajemen
\textit{metadata} foto (\textit{path}, ID) \\

4 & Visualisasi & Matplotlib &
\textit{Library} visualisasi untuk menampilkan hasil pencarian dan pengukuran \\

5 & Citra Digital & OpenCV &
\textit{Library} utama untuk pemrosesan citra digital, digunakan pada tahap \textit{face detection}
dan \textit{face alignment} \\

6 & \textit{Deep Learning} & TensorFlow / PyTorch &
\textit{Framework} untuk membangun dan menjalankan model CNN guna ekstraksi fitur wajah
(\textit{face representation}) \\

7 & \textit{Machine Learning} & Scikit-learn &
Digunakan untuk metode komparasi holistik seperti PCA serta perhitungan metrik evaluasi
(\textit{confusion matrix}) \\

8 & \textit{Integrated Development Environment} & Visual Studio Code / Jupyter Notebook &
Lingkungan pengembangan untuk penulisan kode, pengujian modul per tahap, dan eksperimen algoritma \\

9 & \textit{Version Control} & Git, GitHub &
Digunakan untuk manajemen versi kode dan penyimpanan repositori proyek secara daring \\

10 & Komputasi Awan & Google Colab (GPU \textit{Enabled}) &
Layanan komputasi awan yang menyediakan GPU, untuk mempercepat proses perhitungan fitur wajah jika perangkat lokal tidak memadai \\

11 & Dataset & Dataset Foto Pribadi &
Dataset sebanyak 1.000 foto dengan variasi pose dan pencahayaan, termasuk foto \textit{selfie} pengguna
sebagai data \textit{query} untuk \textit{one-shot learning} \\

\end{longtable}
