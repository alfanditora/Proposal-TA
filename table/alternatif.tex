\begin{table}[H]
  \begin{tabular}{ | p{2cm} | p{4cm} | p{4cm} | p{2cm} | }
    \hline
    \textbf{Alternatif Solusi} & \textbf{Kelebihan} & \textbf{Kekurangan} & \textbf{Relevansi} \\ \hline
    
    1. Berbasis Konten Citra (CBIR) 
    & Dapat diterapkan tanpa memerlukan anotasi manual. 
    & Hanya mengukur kesamaan visual secara umum, tidak dapat mengukur kesamaan identitas wajah pengguna. Rentan terhadap foto yang berbeda orang tetapi memiliki latar atau pencahayaan serupa. 
    & Rendah \\ \hline
    
    2. Berbasis Atribut Semantik 
    & Mampu mempersempit hasil pencarian berdasarkan ciri kontekstual (misalnya, di pantai, memakai kacamata). 
    & Tidak mampu memastikan identitas pengguna secara biometrik karena fokusnya bukan pada wajah melainkan konteks visual. 
    & Rendah \\ \hline
    
    3. Berbasis \textit{Face Recognition}
    & Kemampuan identifikasi yang spesifik terhadap individu. Tingkat akurasi tinggi dan kemudahan penerapan pada \textit{dataset} berskala besar. Mendukung paradigma \textit{one-shot query}. 
    & Memerlukan komputasi yang lebih tinggi dibandingkan pendekatan lain. Kinerja dapat menurun saat dihadapkan pada foto dengan variasi ekstrem (pencahayaan, pose, oklusi). 
    & Tinggi (Solusi Pilihan) \\ \hline

  \end{tabular}
\caption{Perbandingan Peluang Teknologi}
\label{tbl:alternatif_solusi}
\end{table}
