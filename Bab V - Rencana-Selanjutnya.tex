% ==========================================
% BAB V RENCANA SELANJUTNYA
% ==========================================
\chapter{RENCANA SELANJUTNYA}
\label{chap:rencana-selanjutnya}

\section{Rencana Implementasi}
Bagian ini akan menjelaskan mengenai rencana implementasi untuk kedepannya. Terdapat tiga sub bagian yang akan menjelaskan \textit{timeline} implementasi, alat yang dibutuhkan, dan analisis biaya implementasi.

\subsection{\textit{Timeline} Implementasi}
Implementasi dalam tugas akhir ini dibagi lima tahap utama dengan \textit{timeline} selama 11 minggu. \textit{Timeline} ini disusun dengan mempertimbangkan proses antar setiap tahapnya dan viabilitas dalam jadwal semester. Detail \textit{timeline} rencana implementasi ditunjukkan pada Tabel \ref{tbl:timeline}.

\begin{longtable}{@{\extracolsep{\fill}}
    >{\raggedright\arraybackslash}p{2cm}
    >{\raggedright\arraybackslash}p{3cm}
    >{\raggedright\arraybackslash}p{7cm}}
\caption{Timeline Rencana Implementasi}\label{tbl:timeline} \\
\midrule
\textbf{Minggu} & \textbf{Tahap} & \textbf{Kegiatan} \\
\midrule
\endfirsthead

\caption[]{Timeline Rencana Implementasi (lanjutan)} \\
\midrule
\textbf{Minggu} & \textbf{Tahap} & \textbf{Kegiatan} \\
\midrule
\endhead

\midrule
\multicolumn{3}{r}{\textit{Bersambung ke halaman berikutnya}} \\
\endfoot

\midrule
\endlastfoot

1--2 & Persiapan \textit{Dataset} (\textit{Data Preparation}) &
\begin{itemize}
    \item Pengumpulan \textit{dataset} foto untuk skenario pengujian;
    \item Pelabelan (\textit{annotation}) data secara manual untuk membuat \textit{ground truth}.
\end{itemize}
\\

3--5 & Eksplorasi \& Implementasi Algoritma &
\begin{itemize}
    \item Implementasi 5 kategori metode \textit{face representation}: Geometri, Holistik, Berbasis Fitur, Hibrida, dan \textit{Deep Learning};
    \item Pembuatan modul \textit{face detection}, \textit{alignment}, dan \textit{matching};
    \item Pengukuran awal untuk memilih algoritma terbaik berdasarkan kompleksitas model.
\end{itemize}
\\

6--7 & Integrasi Sistem (\textit{System Construction}) &
\begin{itemize}
    \item Pembangunan modul \textit{database} untuk menyimpan \textit{embedding} foto;
    \item Implementasi alur sesuai desain sistem \textit{to-be}.
\end{itemize}
\\

8--9 & Pengujian \& Evaluasi (\textit{Testing}) &
\begin{itemize}
    \item Pengujian performa berdasarkan parameter keberhasilan;
    \item Evaluasi sistem berdasarkan kebutuhan non-fungsional.
\end{itemize}
\\

10--11 & Dokumentasi \& Pelaporan &
\begin{itemize}
    \item Analisis hasil perbandingan antar algoritma;
    \item Penulisan laporan Tugas Akhir lengkap;
    \item Persiapan materi presentasi sidang akhir.
\end{itemize}
\\

\end{longtable}


\subsection{Alat yang Dibutuhkan}

Untuk mendukung tahapan implementasi yang sudah direncakan, diperlukan alat yang berupa perangkat lunak dan perangkat keras yang sesuai. Alat yang dibutuhkan dalam implementasi ditunjukkan pada Tabel \ref{tbl:alat_bahan}.

\begin{longtable}{@{\extracolsep{\fill}} l p{3cm} p{3cm} p{6cm}}
\caption{Alat dan Bahan Penelitian}\label{tbl:alat_bahan} \\
\midrule
\textbf{No.} & \textbf{Kategori} & \textbf{Alat/Bahan} & \textbf{Keterangan} \\
\midrule
\endfirsthead

\caption[]{Alat dan Bahan Penelitian (lanjutan)} \\
\midrule
\textbf{No.} & \textbf{Kategori} & \textbf{Alat/Bahan} & \textbf{Keterangan} \\
\midrule
\endhead

\midrule
\multicolumn{4}{r}{\textit{Bersambung ke halaman berikutnya}} \\
\endfoot

\midrule
\endlastfoot

1 & Perangkat Keras & Laptop (\textit{Workstation}) &
Laptop dengan spesifikasi CPU Intel Core i7, RAM 16 GB, dan SSD yang digunakan untuk pengembangan kode lokal dan dokumentasi \\

2 & Bahasa Pemrograman & Python 3.8+ &
Bahasa pemrograman utama untuk implementasi sistem, dipilih karena dukungan ekosistem
\textit{deep learning} dan pemrosesan citra yang lengkap \\

3 & Data & NumPy, Pandas &
NumPy digunakan untuk operasi matriks pada data citra, sedangkan Pandas untuk manajemen
\textit{metadata} foto (\textit{path}, ID) \\

4 & Visualisasi & Matplotlib &
\textit{Library} visualisasi untuk menampilkan hasil pencarian dan pengukuran \\

5 & Citra Digital & OpenCV &
\textit{Library} utama untuk pemrosesan citra digital, digunakan pada tahap \textit{face detection}
dan \textit{face alignment} \\

6 & \textit{Deep Learning} & TensorFlow / PyTorch &
\textit{Framework} untuk membangun dan menjalankan model CNN guna ekstraksi fitur wajah
(\textit{face representation}) \\

7 & \textit{Machine Learning} & Scikit-learn &
Digunakan untuk metode komparasi holistik seperti PCA serta perhitungan metrik evaluasi
(\textit{confusion matrix}) \\

8 & \textit{Integrated Development Environment} & Visual Studio Code / Jupyter Notebook &
Lingkungan pengembangan untuk penulisan kode, pengujian modul per tahap, dan eksperimen algoritma \\

9 & \textit{Version Control} & Git, GitHub &
Digunakan untuk manajemen versi kode dan penyimpanan repositori proyek secara daring \\

10 & Komputasi Awan & Google Colab (GPU \textit{Enabled}) &
Layanan komputasi awan yang menyediakan GPU, untuk mempercepat proses perhitungan fitur wajah jika perangkat lokal tidak memadai \\

11 & Dataset & Dataset Foto Pribadi &
Dataset sebanyak 1.000 foto dengan variasi pose dan pencahayaan, termasuk foto \textit{selfie} pengguna
sebagai data \textit{query} untuk \textit{one-shot learning} \\

\end{longtable}


\subsection{Biaya yang Dibutuhkan}

Analisis biaya pengembangan dilakukan untuk mengestimasi kebutuhan dana selama proses penelitian dan implementasi sistem berlangsung (estimasi durasi 3 bulan). Rincian estimasi biaya ditunjukkan pada Tabel \ref{tbl:biaya}.

\begin{table}[H]
  \centering
  \begin{tabular}{ | p{1cm} | p{4cm} | p{3cm} | p{4cm} | }
    \hline
    \textbf{No} & \textbf{Komponen} & \textbf{Estimasi Biaya (IDR)} & \textbf{Keterangan} \\ \hline

    1 & Layanan \textit{Cloud Computing} (GPU) &
    Rp 300.000 &
    Langganan Google Colab Pro untuk 2 bulan (Fase minggu 3--7) \\ \hline

    2 & Kuota Internet &
    Rp 0 &
    Unduh \textit{dataset}, \textit{library}, dan referensi jurnal \\ \hline

    3 & Administrasi \& Pelaporan &
    Rp 250.000 &
    Cetak proposal, laporan, dan penjilidan \\ \hline

    % Total Estimasi
    \multicolumn{2}{|c|}{\textbf{Total Estimasi}} &
    \multicolumn{2}{|l|}{\textbf{Rp 550.000}} \\ \hline

  \end{tabular}
\caption{Rencana Anggaran Biaya}
\label{tbl:biaya}
\end{table}


\section{Rencana Pengujian Sistem}

Bagian ini akan menjelaskan mengenai rencana pengujian sistem yang akan dikembangkan. Bagian ini berisi penjelasan metode pengujian dan parameter keberhasilan yang digunakan.

\subsection{Metode Pengujian}

Metode pengujian yang digunakan untuk menguji sistem yang dikembangkan memiliki dua tujuan, yaitu menguji fungsionalitas dan menguji performa. Berikut empat pengujian yang akan dilakukan pada tugas sistem yang dikembangkan di tugas akhir ini:

\begin{enumerate}
    \item Pengujian Fungsional \\
    Pengujian ini dilakukan untuk memastika sistem dapat menerima \textit{input} folder, memproses foto \textit{selfie}, dan menghasilkan keluaran foto pribadi tanpa terjadi kegagalan sistem.
    \item Pengujian Akurasi \\
    Pengujian ini dilakukan untuk mengukur akurasi sistem dalam menyaring foto pribadi dengan membandingkannya pada \textit{ground truth} yang ada.
    \item Pengujian Performa \\
    Pengujian ini dilakukan untuk menghitung waktu rata-rata yang dibutuhkan sistem untuk memproses masukan foto dari pengguna.
    \item Pengujian Stabilitas \\
    Pengujian ini dilakukan untuk menguji ketahanan sistem dengan masukan yang rusak untuk memastikan sistem dapat tetap berjalan. 
\end{enumerate}


\subsection{Parameter Keberhasilan}

Parameter keberhasilan diformulasikan dengan menurunkan dari non-fungsional \textit{requirement}. Sistem dinyatakan berhasil jika memenuhi indikator yang ditunjukkan pada Tabel \ref{tbl:indikator keberhasilan}.

\begin{table}[H]
  \centering
  \begin{tabular}{ | p{1cm} | p{12cm} | }
    \hline
    \textbf{No} & \textbf{Indikator Keberhasilan} \\ \hline

    1 & \textbf{Akurasi Minimal 85\%}: 
    Hasil perhitungan \textit{true positive} dan \textit{true negative} harus menghasilkan skor akurasi di atas 85\%. 
    Angka ini menjadi batas toleransi agar sistem layak digunakan menggantikan pencarian manual. \\ \hline

    2 & \textbf{Waktu Komputasi < 1 Detik/Foto}:
    Sistem harus mampu melakukan deteksi, ekstraksi fitur, dan pencocokan dengan kecepatan rata-rata di bawah 1 detik per foto pada \textit{dataset} pengujian. \\ \hline

    3 & \textbf{Skalabilitas Kapasitas}:
    Sistem mampu memproses \textit{dataset} sebanyak 1.000 foto dalam satu kali eksekusi (\textit{batch processing}) tanpa mengalami kehabisan memori (\textit{Out of Memory}). \\ \hline

    4 & \textbf{Stabilitas Tanpa Crash}:
    Sistem harus memiliki tingkat keberhasilan eksekusi 100\% (tidak \textit{force close}) meskipun terdapat \textit{file} tidak valid atau foto tanpa wajah dalam \textit{dataset} \textit{input}. \\ \hline

  \end{tabular}
\caption{Indikator Keberhasilan}
\label{tbl:indikator keberhasilan}
\end{table}


\section{Analisis Risiko}

Bagian ini akan menjelaskan risiko yang mungkin dapat terjadi dalam pengerjaan tugas akhir ini. Risiko yang teridentifikasi akan dijelaskan penyebab, probabilitas, dampak, dan stategi mitigasinya. Analisis risiko dan mitigasi ditunjukkan pada Tabel \ref{tbl:risiko_mitigasi}.

\begin{longtable}{@{\extracolsep{\fill}} p{0.8cm} p{3cm} p{3cm} p{2.6cm} p{2.6cm}}
\caption{Analisis Risiko dan Strategi Mitigasi} \label{tbl:risiko_mitigasi} \\
\midrule
\textbf{No.} & \textbf{Risiko Potensial} & \textbf{Penyebab Utama} & \textbf{Dampak} & \textbf{Strategi Mitigasi} \\
\midrule
\endfirsthead

\caption[]{Analisis Risiko dan Strategi Mitigasi (lanjutan)} \\
\midrule
\textbf{No.} & \textbf{Risiko Potensial} & \textbf{Penyebab Utama} & \textbf{Dampak} & \textbf{Strategi Mitigasi} \\
\midrule
\endhead

\midrule
\multicolumn{5}{r}{\textit{Bersambung ke halaman berikutnya}} \\
\endfoot

\midrule
\endlastfoot

1 &
Kegagalan Deteksi Wajah &
Resolusi foto input terlalu rendah atau wajah terhalang oklusi (masker/kacamata) &
Akurasi sistem turun drastis (\textit{False Negative} tinggi) &
Melakukan \textit{upscaling} citra otomatis sebelum deteksi dan menurunkan \textit{confidence threshold} \\

2 &
\textit{Resource Exhaustion} (OOM) &
Dataset 1.000 foto melebihi kapasitas memori saat pemrosesan &
Proses berhenti mendadak (crash) atau perangkat menjadi tidak responsif &
Menggunakan \textit{batch processing} dan melakukan \textit{resize} foto ke resolusi standar (800px) \\

3 &
Format Data Korup &
Kesalahan unduhan atau format file tidak didukung (RAW/HEIC) &
Sistem \textit{error} saat membaca \textit{file} dan iterasi terhenti &
Menambahkan \textit{error handling} untuk melewati \textit{file} rusak dan mencatatnya dalam \textit{log} \\

4 &
Hambatan Komunikasi Pembimbing &
Jadwal dosen padat atau sulit mencocokkan waktu bimbingan &
Keterlambatan umpan balik dan risiko revisi yang besar &
Menyepakati jadwal bimbingan rutin dan mengirim laporan progres secara berkala \\

5 &
Timeline Tidak Realistis &
Estimasi waktu terlalu optimistis tanpa memperhitungkan \textit{debugging} &
Pengerjaan molor, waktu pengujian berkurang, risiko penundaan sidang &
Menerapkan \textit{buffer time} 1 minggu per fase dan memprioritaskan fitur utama \\

\end{longtable}
