% ==========================================
% BAB V RENCANA SELANJUTNYA
% ==========================================
\chapter{RENCANA SELANJUTNYA}
\label{chap:rencana-selanjutnya}

\section{Rencana Implementasi}
Bagian ini akan menjelaskan mengenai rencana implementasi untuk kedepannya. Terdapat tiga sub bagian yang akan menjelaskan \textit{timeline} implementasi, alat yang dibutuhkan, dan analisis biaya implementasi.

\subsection{\textit{Timeline} Implementasi}
Implementasi dalam tugas akhir ini dibagi lima tahap utama dengan \textit{timeline} selama 11 minggu. \textit{Timeline} ini disusun dengan mempertimbangkan proses antar setiap tahapnya dan viabilitas dalam jadwal semester. Detail \textit{timeline} rencana implementasi ditunjukkan pada Tabel \ref{tbl:timeline}.

\begin{longtable}{@{\extracolsep{\fill}}
    >{\raggedright\arraybackslash}p{2cm}
    >{\raggedright\arraybackslash}p{3cm}
    >{\raggedright\arraybackslash}p{7cm}}
\caption{Timeline Rencana Implementasi}\label{tbl:timeline} \\
\midrule
\textbf{Minggu} & \textbf{Tahap} & \textbf{Kegiatan} \\
\midrule
\endfirsthead

\caption[]{Timeline Rencana Implementasi (lanjutan)} \\
\midrule
\textbf{Minggu} & \textbf{Tahap} & \textbf{Kegiatan} \\
\midrule
\endhead

\midrule
\multicolumn{3}{r}{\textit{Bersambung ke halaman berikutnya}} \\
\endfoot

\midrule
\endlastfoot

1--2 & Persiapan \textit{Dataset} (\textit{Data Preparation}) &
\begin{itemize}
    \item Pengumpulan \textit{dataset} foto untuk skenario pengujian;
    \item Pelabelan (\textit{annotation}) data secara manual untuk membuat \textit{ground truth}.
\end{itemize}
\\

3--5 & Eksplorasi \& Implementasi Algoritma &
\begin{itemize}
    \item Implementasi 5 kategori metode \textit{face representation}: Geometri, Holistik, Berbasis Fitur, Hibrida, dan \textit{Deep Learning};
    \item Pembuatan modul \textit{face detection}, \textit{alignment}, dan \textit{matching};
    \item Pengukuran awal untuk memilih algoritma terbaik berdasarkan kompleksitas model.
\end{itemize}
\\

6--7 & Integrasi Sistem (\textit{System Construction}) &
\begin{itemize}
    \item Pembangunan modul \textit{database} untuk menyimpan \textit{embedding} foto;
    \item Implementasi alur sesuai desain sistem \textit{to-be}.
\end{itemize}
\\

8--9 & Pengujian \& Evaluasi (\textit{Testing}) &
\begin{itemize}
    \item Pengujian performa berdasarkan parameter keberhasilan;
    \item Evaluasi sistem berdasarkan kebutuhan non-fungsional.
\end{itemize}
\\

10--11 & Dokumentasi \& Pelaporan &
\begin{itemize}
    \item Analisis hasil perbandingan antar algoritma;
    \item Penulisan laporan Tugas Akhir lengkap;
    \item Persiapan materi presentasi sidang akhir.
\end{itemize}
\\

\end{longtable}


\subsection{Alat yang Dibutuhkan}

Untuk mendukung tahapan implementasi yang sudah direncakan, diperlukan alat yang berupa perangkat lunak dan perangkat keras yang sesuai. Alat yang dibutuhkan dalam implementasi ditunjukkan pada Tabel \ref{tbl:alat_bahan}.

\begin{longtable}{@{\extracolsep{\fill}} l p{3cm} p{3cm} p{6cm}}
\caption{Alat dan Bahan Penelitian}\label{tbl:alat_bahan} \\
\midrule
\textbf{No.} & \textbf{Kategori} & \textbf{Alat/Bahan} & \textbf{Keterangan} \\
\midrule
\endfirsthead

\caption[]{Alat dan Bahan Penelitian (lanjutan)} \\
\midrule
\textbf{No.} & \textbf{Kategori} & \textbf{Alat/Bahan} & \textbf{Keterangan} \\
\midrule
\endhead

\midrule
\multicolumn{4}{r}{\textit{Bersambung ke halaman berikutnya}} \\
\endfoot

\midrule
\endlastfoot

1 & Perangkat Keras & Laptop (\textit{Workstation}) &
Laptop dengan spesifikasi CPU Intel Core i7, RAM 16 GB, dan SSD yang digunakan untuk pengembangan kode lokal dan dokumentasi \\

2 & Bahasa Pemrograman & Python 3.8+ &
Bahasa pemrograman utama untuk implementasi sistem, dipilih karena dukungan ekosistem
\textit{deep learning} dan pemrosesan citra yang lengkap \\

3 & Data & NumPy, Pandas &
NumPy digunakan untuk operasi matriks pada data citra, sedangkan Pandas untuk manajemen
\textit{metadata} foto (\textit{path}, ID) \\

4 & Visualisasi & Matplotlib &
\textit{Library} visualisasi untuk menampilkan hasil pencarian dan pengukuran \\

5 & Citra Digital & OpenCV &
\textit{Library} utama untuk pemrosesan citra digital, digunakan pada tahap \textit{face detection}
dan \textit{face alignment} \\

6 & \textit{Deep Learning} & TensorFlow / PyTorch &
\textit{Framework} untuk membangun dan menjalankan model CNN guna ekstraksi fitur wajah
(\textit{face representation}) \\

7 & \textit{Machine Learning} & Scikit-learn &
Digunakan untuk metode komparasi holistik seperti PCA serta perhitungan metrik evaluasi
(\textit{confusion matrix}) \\

8 & \textit{Integrated Development Environment} & Visual Studio Code / Jupyter Notebook &
Lingkungan pengembangan untuk penulisan kode, pengujian modul per tahap, dan eksperimen algoritma \\

9 & \textit{Version Control} & Git, GitHub &
Digunakan untuk manajemen versi kode dan penyimpanan repositori proyek secara daring \\

10 & Komputasi Awan & Google Colab (GPU \textit{Enabled}) &
Layanan komputasi awan yang menyediakan GPU, untuk mempercepat proses perhitungan fitur wajah jika perangkat lokal tidak memadai \\

11 & Dataset & Dataset Foto Pribadi &
Dataset sebanyak 1.000 foto dengan variasi pose dan pencahayaan, termasuk foto \textit{selfie} pengguna
sebagai data \textit{query} untuk \textit{one-shot learning} \\

\end{longtable}


\subsection{Biaya yang Dibutuhkan}

Analisis biaya pengembangan dilakukan untuk mengestimasi kebutuhan dana selama proses penelitian dan implementasi sistem berlangsung (estimasi durasi 3 bulan). Rincian estimasi biaya ditunjukkan pada Tabel \ref{tbl:biaya}.

\begin{table}[H]
  \centering
  \begin{tabular}{ | p{1cm} | p{4cm} | p{3cm} | p{4cm} | }
    \hline
    \textbf{No} & \textbf{Komponen} & \textbf{Estimasi Biaya (IDR)} & \textbf{Keterangan} \\ \hline

    1 & Layanan \textit{Cloud Computing} (GPU) &
    Rp 300.000 &
    Langganan Google Colab Pro untuk 2 bulan (Fase minggu 3--7) \\ \hline

    2 & Kuota Internet &
    Rp 0 &
    Unduh \textit{dataset}, \textit{library}, dan referensi jurnal \\ \hline

    3 & Administrasi \& Pelaporan &
    Rp 250.000 &
    Cetak proposal, laporan, dan penjilidan \\ \hline

    % Total Estimasi
    \multicolumn{2}{|c|}{\textbf{Total Estimasi}} &
    \multicolumn{2}{|l|}{\textbf{Rp 550.000}} \\ \hline

  \end{tabular}
\caption{Rencana Anggaran Biaya}
\label{tbl:biaya}
\end{table}


\section{Rencana Pengujian Sistem}

Bagian ini akan menjelaskan mengenai rencana pengujian sistem yang akan dikembangkan. Bagian ini berisi penjelasan metode pengujian dan parameter keberhasilan yang digunakan.

\subsection{Metode Pengujian}



\subsection{Parameter Keberhasilan}

\section{Analisis Risiko}